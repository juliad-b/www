\section{Gilmour}

\subsection{FIRST GENERATION]

The first person in our direct line that we know of is \uppercase{James Gilmour} who was born in 1745;  he married \uppercase{Janet Akred} on 15 December 1765 in Dundonald, Ayrshire. They had at least one son, \uppercase{Joseph Gilmour} (1774 - 1837) who was \uppercase{Joan Nyria Hancox}'s great great grandfather.

\uppercase{Joseph Gilmour} married \uppercase{Mary Boyd Clark} on 8 July 1797 in Fenwick, Ayrshire, Scotland. They had seven children: Elizabeth Gilmour, Joseph Gilmour, James Gilmour, Allan Gilmour, Andrew Gilmour, Robert Gilmour, and \uppercase{Boyd Gilmour}, who was \uppercase{Joan Nyria Hancox}'s great grandfather.

Their only daughter, \textsc{Elizabeth Gilmour} was born on 4 December 1797 in Kilmarnock, Ayrshire. She married John Falconer (a colliery worker) on 8 December 1821 in Muirkirk, Ayrshire and they lived in Loudoun, Ayrshire. They had nine children: Joseph Falconer (b.1821), Archibald Falconer (b.1822), John Falconer (b.1824), Mary Falconer (b.1827), Thomas Falconer (b.1829), Mary Falconer (b.1831), Janet Falconer (b.1834), Elizabeth Falconer (b.1836) and James Falconer (b.1838).  She died on 30 August 1870 in Kilmarnock, Ayrshire.

Their first son, \textsc{Joseph Gilmour}, was born on 22 November 1802 in Sorn,	Ayrshire.  He married Helen Whittan on 27 July 1838 in Riccarton, Ayrshire and they had four children: Joseph Gilmour (b.1839),  George Gilmour (b.1842), Mary Gilmour (b.1844), and Andrew Gilmour (b.1848). He died on	21 June 1851 in Riccarton.	

Their second son, \textsc{James Gilmour} was born on 9 March 1805 in Kilmarnock, Ayrshire. He married Marion Ross on 26 October 1826 in Muirkirk, Ayrshire. He was a coalmaster. They had five children: Janet Gilmour (b.1827), Joseph Gilmour (b.1829), John Gilmour (b.1832), Allan Gilmour (b.1834) and Alexander Gilmour (b.1836). He died on 26 March 1866 at Hillhead, Kilmarnock.

Their third son, \textsc{Allan Gilmour}, was born on 3 April 1807 in Riccarton, Ayrshire. He married Jean Williamson and they had one son, Allan Gilmour (b.1833 - 1906).  He later married Catherine Campbell on 18 August 1843 in Riccarton, Ayrshire and they had seven children: Mary Gilmour (b.1838), Elizabeth Gilmour (b.1843), Catherine Gilmour (b.1845), Joseph Gilmour (b.1847), Barbara Gilmour (b.1849), Daniel Gilmour (1851 - 1924), and Flora Campbell Gilmour (b.1853).  Allan was a coalmaster and was very prosperous, living at Woodend House, Hurlford, near Kilmarnock.  He died on 22 April 1854 in Hurlford.

The fifth child in this family was \textsc{Andrew Gilmour}, who was born on 5 April 1810 in Riccarton, Ayrshire. He was a colliery overman, and then manager, in Loudoun, Ayrshire and lived in Boyd Street. He possibly married a Mary Groves.  He died 
on 25 March 1874 in Loudoun.

The younest son was \textsc{Robert Gilmour}  who was born on 15 February 1812 in Riccarton, Ayrshire. He married Elizabeth Whittan (possibly sister to his brother's wife Helen) on 30 June 1837 in Kilmarnock, Ayrshire in the Low church and died in April 1841 in Riccarton, age 29.

\subsection{SECOND GENERATION}

\uppercase{Boyd Gilmour} married \uppercase{Jean Dunsmore} on 26 June 1835 in Riccarton, Ayrshire.  They had eight children:  Jean Gilmour, Joseph Gilmour (who died as an infant), Joseph Gilmour, \uppercase{Mary Gilmour} (who was \uppercase{Joan Nyria Hancox}'s grandmother), Marion Gilmour, Boyd Gilmour, Allan Columbia Gilmour and John Gilmour. 

The oldest child in the family was \textsc{Jean Gilmour} who was born on 18 March 1836. She married John Login Sinclair on 17 December 1855 in Kilmarnock, Ayrshire, and they had four sons: John Logan Sinclair (b.1857), Henry Kendall Sinclair (b.1861), 
Boyd Sinclair (b.1865) and Joseph Allan Columbia Gilmour Sinclair (1866 - 1942).  In 1871 she was living with her sons in Glasgow, and by 1881 she was living at Doon Cottage, Rothesay, Bute next door to her brother Allan Columbia.

The next child was \textsc{Joseph Gilmour}, who was born on 5 July 1840. In 1864 he was an engine fitter in Kilmarnock. He married Margaret Baird on 27 July 1864 in Kilmarnock. They had six children: Jane Gilmour (b.1865), Boyd Gilmour (b.1867), Sarah Gilmour, Alice Gilmour, Joseph Allan Columbia Gilmour (b.1874), and Maud Gilmour. They emigrated to the United States and in 1910 were living in Port Carbon, Schuylkill, Pennsylvania.

Their third daughter was \textsc{Marion Gilmour} who was born on 1 January 1847 in Riccarton, Ayrshire. She was a draper in Saltcoats, Ayrshire and never married. She died on 19 November 1928 in Ardrossan, Ayrshire, of myocarditis and cardiac failure.

Their next child was \textsc{Boyd Gilmour}, who was born on 11 January 1849 in Riccarton, Ayrshire. He was an engineer/engine fitter in Kilmarnock. He married Annie Beattie on 6 October 1871 in Kilmaurs, Ayrshire and they had one son, Boyd Gilmour (1873 - 1934). The family emigrated to the United States in 1882 on the Erin and are listed on the Passenger Manifest as coming from Liverpool via Queenstown, and going to the District of New York, Port of New York. 1882, 17th. June. (`` SS Erin (National Line - British flag) Passenger List (Steerage) to New York, United States, having sailed from Liverpool, England to New York via Queenstown, (Co. Cork), Ireland: Boyd Gilmour (3 age incomplete), Male, Farmer; Annie Gilmour (20 - looks like), Female, Wife; Boyd Gilmour (9 - looks like), Male, Child.''

The next son, \textsc{Allan Columbia Gilmour} was born on 20 June 1851: recorded as being born at sea in the 1881 census, he was later listed as being born on the Columbia River, Oregon Territory (he and his cousin were both born on the journey out to Vancouver Island). In 1871, when the family had returned to Scotland, he was a second mate, and by 1874, he was a first mate (certificate No.13025). He married Christine Knox (whose father was an innkeeper) on 17 March 1876 in Monkton and Prestwick, Ayrshire and they lived at Portland Terrace, Troon. By 1881, they were living at Dove Cottage, Rothesay, Bute and he was a coal merchant.   When his daughter Marion was born, he is still recorded as being a seaman in the merchant service, but the following year the 1881 census shows him as being a coal merchant. They lost two infant sons, one on 24 February 1869 and one at 7 months, on 23 March 1882 (also Allan Columbia, born 31 August 1881) and also had two daughters, Jeanie Dunsmuir Knox Gilmour (b.1877) and Marion Gilmour (b.1880). The family emigrated to America in 1881, arriving on the Steamship "Prussian" in Philadelphia on 13 May.

Their youngest child was \textsc{John Gilmour} who was born on 24 December 1854 in Victoria, Vancouver Island, Canada, and who died of 'water on the head' on 31 January 1856 in Kilmarnock, Ayrshire. He was buried at the St Andrews Burying Ground.



  

