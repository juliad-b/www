\section{CROSKERY}

The first Croskery that we know of in this family line was \uppercase{Hugh Croskery}, who was born in Downpatrick, Co. Down, Ireland in 1803.  He married Charlotte Wallace Brown on 9 May 1834 in Ballynahinch,  and they had eight children: Hugh Croskery, Ann Croskery, Alexander Brown Croskery, Albert James Croskery, Horatio Collingwood Croskery, Frederick C. Croskery, \uppercase{Captain Samuel Maxwell West Croskery} (who was \uppercase{Joan Nyria Hancox}'s maternal grandfather, and Wallace Brown Croskery.

Their eldest son was \textsc{Hugh Croskery},  who was born on 13 January 1835 in Downpatrick, County Down. In June 1856 he was an Acting naval assistant surgeon for the British Navy, and he emigrated to Chapelton, Jamaica in 1857.  He married Charlianna Hall on 19 July 1859. By 1875 he was working in British Guiana as a Physician and Missionary, where he died in 1886. There is a short piece about him in `The Kingston Roundabout'on 1 November 1966:
\begin{quotation}
The Reverend Dr Hugh Croskery, MRCSI, JP, RN, studied medicine and received a medical degree. Thereafter, he served as a Naval Surgeon in the British Navy. Later, in 1859, Hugh moved to Jamaica where he took the position of a District Medical Officer. After studying for the ministry, he was ordained an Anglican priest and served under his father-in-law, the Rev. Charles Hall. He wrote ``The gospel of the kingdom, a vade-mecum of texts and prayers of intercession'' (1877) which has an introduction by the Bishop of Jamaica, an excerpt of which reads: ``The district of Chapelton, in Jamaica, lies amid the interior mountains of a tropical island, over which the residences of the peasantry are scattered far and wide. The Rector is no longer young. His most remote station is at a distance of more than twenty miles from his residence, to be traversed, for the most part, by a bridle path, narrow, steep and rugged. The Rev. Hugh has but one servant accompanying him, and they travel by donkey.'' At his death, Hugh was buried in the family cemetery at Half Way Tree, Kingston, Jamaica.
\end{quotation}
In the following excerpt, H.P. Jacobs gave an account about Hugh in a broadcast on Radio Jamaica on 1 November 1966:
\begin{quotation}
There is no trace now of the old house at 85 East Street, where today you will find a new building occupied by Reckitt \& Coleman (Overseas) Ltd.  But the old building was occupied nearly a century ago by someone whose name will perhaps  
have an oddly familiar ring for you. The Rev. Hugh Croskery died at the age of 51, but almost everything about his not very long life was unusual, including his death. Croskery was an Irishman, an Ulsterman, born in 1835.  He started as a naval surgeon, of all things, and on the strength of a Medical degree became a District Medical Officer in Jamaica. That was as early as 1857, and he stayed many years at Chapelton, where he married the daughter of the Rev. Charles Hall, who was at Chapelton for a quarter of a century. In 1871 he became an Anglican deacon and served at Chapelton as curate to his father-in-law.  He continued to practice as a doctor and is about the nearest I can find to a medical missionary in Jamaica. He wrote a book, ``The Gospel of the Kingdom''.
 \end{quotation}
    
Their second child and only daughter, \textsc{Ann Croskery}, was born on 16 May 1836 in Downpatrick, County Down, and died in 1931. 

Their second son \textsc{Alexander Brown Croskery} was born on 10 March 1838 in	Downpatrick, County Down, and emigrated to New Zealand where he was an accountant in Wellington, New Zealand and also worked as an auctioneer.  He married Mary Ann Mortimer Thomson (1850-1925) and they had four children: Alexander Wellington Croskery (1878-1952), who had a notable career as a union reformer and is listed in the Dictionary of New Zealand Biography,  William Hugh Croskery (1881-1948), Victor John Croskery (1886-1964) and A/Eileen Charlotte Croskery (1890-1969). He died on 3 April 1897 and is buried at the Karori Cemetry in Wellington.

Their third son, \textsc{Albert James Croskery} was born on	14 July 1840 in	Downpatrick, County Down, and died on 17 January 1865 in a shipwreck of the \textit{Columbian}, off Ushant, Brittany, France, an island in the English Channel. There is a short article about this in the Old Mersey Times, in 1865:
\begin{quotation}
Mr CROSKERRY joined in October or November 1863, as 2nd officer. Witness had subsequently watched his course closely, and recommended him to pass as 1st master, and having done so he was put on board the COLOMBIAN. Not only from what the witness had seen, but, from the reports, he had every reason to believe that CROSKERRY was a perfectly efficient officer. He had a master's certificate. The other officers had good recommendations.
The captain took an observation and said the ship was near Ushant, the vessel was then driving in shore, and in a short time struck on the rocks, she drifted off again and half an hour afterwards went down. When the ship was going down the captain came to the lobby, and witness assisted him off with his boots, the captain went to his room as the ship was sinking, he was not sober then. After the captain's boots were taken off he slipped on the deck, and did not appear to make any effort to save himself, he could not, the man was completely gone. The ship went down sucking all hands with her except the witness and two men, they got on to a pigstye and were afterwards picked up by the pilot boat.
\end{quotation}

Their fourth son was \textsc{Horatio Collingwood Croskery}, who was born on	21 July 1842 in Downpatrick, County Down. He married Annie Arnold Hutton in 1874 and they had five children: Margaret Croskery, Charlotte Wallace Croskery (b.abt. 1883), 
Catherine Elizabeth Croskery (b.abt 1885), Winifred Croskery (1888-1967) and Horatio Collingwood Croskery (b.1891).  Horatio was a General Merchant and had a warehouse at 16 Market Street, Downpatrick, and yards in Market Street and Quoile Quay: the departments were Family Groceries, wines and Spirits, Home and Foreign Provisions, Farm and Garden Seeds, Artificial Manures, Iron and Coal. He owned a stable and yard, and either owned or leased (from Major Robert Wallace) Nos. 5,6, 7 and 86, Irish Street, Downpatrick. In the 1901 Census he is noted as being a Publican and Grocer, but by 1911 he listed himself as being a Farmer at 1 Ballywarren. The family were Unitarian and could all read and write. He also owned two ships, as follows:  the schooners Glide and Nelson (No. 37182 \textit{Glide}; registered Belfast; a schooner of 64 tons; built Bay of Verte New Brunswick 1861; owned by Horatio Croskery of Downpatrick; Lloyd’s- builder Edward Gooden; dimensions 70.7 x 21.5 x 8.3; re-registered Belfast Nov 1900 upon restoration to seaworthiness and all Belfast owned; register closed March1915- being broken up. And also, No. 64460 \textit{Nelson},registered Belfast; a schooner of 134 tons; built Coverdale 1870; owned by Horatio Croskery of Downpatrick; Lloyd’s- builder Wright; dimensions 94 x25.8 x 9.4 Board of Trade 2887/88- on 7 Aug 1887 on Strangford to Ardrossan, in ballast, about w mile north of Corsewell Point, Firth of Clyde, sank following collision with the steamer Ayrshire of Belfast; crew and a passenger all saved.)   Horatio died on 5 June 1929, and in his will he left £7,445-8s-0d; the testator was his son Horatio Collingwood, Coal Merchant. 

Their next son was \textsc{Frederick C. Croskery}, born on 13 March 1845 in Downpatrick, County Down. Nothing more is known of him.

Their youngest son was \textsc{Wallace Brown Croskery}, born on 6 February 1851 in Downpatrick, County Down.  By 1881 he was living in Marylebone, Middlesex at 1 Dorset Square as a Medical Assistant, and later moved to Eckington, Derbyshire, where he worked as a Physician and Surgeon  (he is in the medical register of 1913, shown as living at Springfield House Eckington Derbyshire: Registered 29 jan 1876 as a ``Lic r coll surg ire 1875, lic lic midwife 1879, K Q Royl coll phys Irel 1876'') until his death on 27 April 1926 in Chesterfield, Derbyshire. He never married.  His will stated: ``My practice is not to be sold and no debts are to be collected.''  He also stipulated that fifty pounds was to be allocated to the care and upkeep of his parrot and his dog.



