\section{Croskery}

\subsection{FIRST GENERATION}

The first Croskery that we know of in this family line was Hugh Croskery, who was born in Downpatrick, Co. Down, Ireland in 1803.  He married Charlotte Wallace Brown on 9 May 1834 in Ballynahinch,  and they had eight children: Hugh Croskery (1835 - 1886), Ann Croskery(1836 - 1931), Alexander Brown Croskery (1838 - 1897), Albert James Croskery (1840 - 1865), Horatio Collingwood Croskery (1842 - 1929), Frederick C. Croskery (1845 -), Captain Samuel Maxwell West Croskery (1847 - 1933) who was Joan Nyria Denton Barker's (nee Hancox) grandfather, and Wallace Brown Croskery (1851 - 1926).
Hugh was a spirit and porter dealer and grocer in Scotch Street, Downpatrick, in 1846, and in Market Street in 1850, and also a Publican in Scotch Street in 1852. He was also a ship owner and possibly owned a mine, and a farm. He died after 1897.

Their first son was \textsc{Hugh Croskery},  who was born on 13 January 1835 in	Downpatrick, County Down. In June 1856 he is recorded as being an Acting naval assistant surgeon for the British Navy, and emigrated to Chapelton, Jamaica in 1857.  He married Charlianna Hall on	19 July 1859. By 1875 he was working in	British Guiana as a Physician and Missionary where he died in 1886. There is a short piece about him in "The Kingston Roundabout" on 1 November 1966:

\begin{quotation}

The Reverend Dr Hugh Croskery, MRCSI, JP, RN, studied medicine and received a medical degree. Thereafter, he served as a Naval Surgeon in the British Navy. Later, in 1859, Hugh moved to Jamaica where he took the position of a District Medical Officer. After studying for the ministry, he was ordained an Anglican priest and served under his father-in-law, the Rev. Charles Hall. He wrote 'The gospel of the kingdom, a vade-mecum of texts and prayers of intercession' (1877) which has an introduction by the Bishop of Jamaica, an excerpt of which reads: "The district of Chapelton, in Jamaica, lies amid the interior mountains of a tropical island, over which the residences of the peasantry are scattered far and wide. The Rector is no longer young. His most remote station is at a distance of more than twenty miles from his residence, to be traversed, for the most part, bu a bridle path, narrow, steep and rugged." (The Rev. Hugh had one servant accompanying him, and they travelled by donkey.) At his death, Hugh was buried in the family cemetery at Half Way Tree, Kingston, Jamaica. In the following excerpt, H.P. Jacobs gave an account about Hugh in a broadcast on RJR 

(Radio Jamaica, Ltd.) on 1 November 1966:

    "There is no trace now of the old house at 85 East Street, where today you will  
    find a new building occupied by Reckitt & Coleman (Overseas) Ltd.  But the old building 
    was occupied nearly a century ago by someone whose name will perhaps  
    have an oddly familiar ring for you. 
         "The Rev. Hugh Croskery died at the age of 51, but almost everything about his  
    not very long life was unusual, including his death. Croskery was an Irishman, an  
    Ulsterman, born in 1835.  He started as a naval surgeon, of all things, and on the  
    strength of a Medical degree became a District Medical Officer in Jamaica. That was  
    as early as 1857, and he stayed many years at Chapelton, where he married the  
    daughter of the Rev. Charles Hall, who was at Chapelton for a quarter of a century.   
    In 1871 he became an Anglican deacon and served at Chapelton as curate to his  
    father-in-law.  He continued to practice as a doctor and is about the nearest I can  
    find to a medical missionary in Jamaica. He wrote a book, "The Gospel of the Kingdom".
    \end{quotation}
    
Their second child, \textsc{Ann Croskery} was born on 16 May 1836 in	Downpatrick, County Down, and died in 1931.
Their son \textsc{Alexander Brown Croskery} was born on 10 March 1838 in	Downpatrick, County Down, and emigrated to New Zealand where he was an accountant in Wellington, New Zealand and also worked as an auctioneer. He died on 3 April 1897 and is buried at the Karori Cemetry in Wellington.

Their third son, \textsc{Albert James Croskery} was born on	14 July 1840 in	Downpatrick, County Down, and died on	17 January 1865 in a shipwreck of the "Columbian" off Ushant, Brittany, France, an island in the English Channel. There is a short article about this in the Old Mersey Times, in 1865:
\begin{quotation}

Mr CROSKERRY joined in October or November 1863, as 2nd officer. Witness had subsequently watched his course closely, and recommended him to pass as 1st master, and having done so he was put on board the COLOMBIAN. Not only from what the witness had seen, but, from the reports, he had every reason to believe that CROSKERRY was a perfectly efficient officer. He had a master's certificate. The other officers had good recommendations.
The captain took an observation and said the ship was near Ushant, the vessel was then driving in shore, and in a short time struck on the rocks, she drifted off again and half an hour afterwards went down. When the ship was going down the captain came to the lobby, and witness assisted him off with his boots, the captain went to his room as the ship was sinking, he was not sober then. After the captain's boots were taken off he slipped on the deck, and did not appear to make any effort to save himself, he could not, the man was completely gone. The ship went down sucking all hands with her except the witness and two men, they got on to a pigstye and were afterwards picked up by the pilot boat.
\end{quotation}






    



