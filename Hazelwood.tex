\section{Hazelwood}

\subsection{FIRST GENERATION}

\uppercase{Thomas Heslewood} married \uppercase{Mary Unknown} and they had at least one son, \uppercase{Moses Hezelwood} (note that the spelling of the famil name changed at that point) but there is no record of any other children.  Moses married \uppercase{Elizabeth Meade} on 22 April 1802 and they had eight children: Mary Hezelwood, \uppercase{Elizabeth Hazelwood} (who was \uppercase{Ralph Munday Denton Barker}'s great grandmother), Isabella Hazelwood, Sarah Hazelwood, Francis Medd Hazelwood, Thomas Hezelwood, Francis Hazelwood and Trufit Mead Hazelwood. 

\textsc{Mary Hezelwood} was born on 19 January 1805 in Whitby, Yorkshire, and christened on 22 January 1805 at St.Mary's,	Whitby. She died on 16 December 1887 at Winchmore Hill, Buckinghamshire, and was buried at Edmonton, Middlesex.

\textsc{Isabella Hazelwood}, known in the family as Aunt Bell, was born on 21 November 1808 in Whitby, and lived in Bathgate with her father until moving to Liverpool, where she lived with her sister Elizabeth and her family at 79 Canning Street. By 1881 she was living with her nephew, \uppercase{Thomas Henry Barker} and his family, at 44 Orrell Park.  She died on 19 December 1882 in Liverpool, and was buried on 24 December 1882 at the Anfield Cemetery. 	

\textsc{Sarah Hazelwood} was born on 25 March 1811 in	Whitby, Yorkshire. Nothing more is known of her life.
\textsc{Francis Mead Hezelwood} was born on 4 February 1813 in Whitby, Yorkshire.

\textsc{Thomas Hezelwood} was born in January 1814 in	Whitby, and christened on 30 January 1814.  Thomas ``left Whitby on the Wednesday the 9th of May 1838 started work at J. Storrs, Mortlake, Surrey on the following Wednesday.'' He wrote a lot of poems, and also collected poems from various publications as well as those by well known writers. One example of his poetry is titled: ``Lines composed on Whitby Cliff''.  He worked as a cabinet maker and was living at Smiths Cottage, Mortlake, Surry in 1841. He died on 2 February 1851 at 6 James Street, Salmons Lane, Stepney and was buried on 7 February at Limehouse Churchyard.

\textsc{Francis Hazelwood} was born on 8 May 1816 in Whitby, Yorkshire.
\textsc{Trufit Mead Hazelwood} was born in Whitby and christened on 16 December 1817.


