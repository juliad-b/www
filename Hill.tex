\section{HILL}

Our earliest record of a Hill in this line was \uppercase{Robert Hill} who was born in High Littleton, Somerset. He married \uppercase{Isabel Greste} and they had at least one son, \uppercase{Robert Hill}. He died about 1607.

\uppercase{Robert Hill} was born in Paulton, Somerset and by 1637 he was living at Paulton House. He married \uppercase{Mary Gullock} on 1 October 1637 in Timsbury, Somerset.  They had two sons: \uppercase{John Hill} and \textsc{James Hill}, of whom nothing is known.

\uppercase{John Hill} was born in Paulton, and he married Elizabeth James.  They had three children: \uppercase{Joseph Hill}, John Hill and Elizabeth Hill.
Their second son, \textsc{John Hill}, lived in Paulton, and he married Mary Anthony. They had four daughters.  He was engaged in mining coal in the area near Paulton.  In 1716 he leased out the mining rights under Grindstone and Allard’s Ham. This was the land between the Hallatrow-Paulton Road and the Cam brook, almost opposite the entrance to Butts Lane:
\begin{quotation}
Extract of Indenture dated 15 Oct 3 George [1716] between John HILL of Hallatrough (1), Mr John PURNELL of High Littleton, George CARTER of Charlton, Kilmersdon \& Henry GREGORY of Paulton (2). Whereby HILL (1) allows (2) to dig for coal etc. in certain lands, namely: Grinstone, Hooked Meade and Alardsham (7 acres) for 21 years from next lady day at £7.15/- p.a. (Sgd) George CARTER, John PURNELL, Henry GREGORY. Witnesses Cha’s STEPHENS, Thomas THRESHER.
\end{quotation}
The manor of Hallatrow, where he lived, remained in the Hill family until the early 18th century. When he died in May 1738,  his Hallatrow estate was left to his four daughters, whose husbands put it up for sale. Amongst the prospective purchasers was Joseph Langton of Newton Park, Newton St. Loe, who commissioned a survey of the estate in 1716, from which it is obvious that coalmining activity was or had been going on before then. Amongst the details were: ``Survey of Hallatrow Farm, 1716 Schedule of land and property including: Coale Pitt Ground Pasture 3 acres Value £2 per acre (1720 in 2 closes) Whole estate 135½ acres let to James Collins @ £90 p.a. For timber \& Coale £100 + 3 tenements \& closes let to others Value of timber, Coale \& Herriotts £10.''
He was buried on 8 May 1738 in Paulton churchyard.
Their only daughter, \textsc{Elizabeth Hill},  was born in Paulton and married Samuel Coombes.

In the next generation, \uppercase{Joseph Hill} was born in Paulton, Somerset and married \uppercase{Mary ?} in 1726 in Paulton, Somerset. They had four children: Joseph Hill, \uppercase{John Hill}, (who was \uppercase{Ralph Mundy Denton Barker}'s great-great-great grandfather), Robert Hill, and Elizabeth Hill.
The eldest son, \textsc{Joseph Hill}, was born in 1727 and married Mary Saunders on 19 July 1752 in Paulton, Somerset.  Presumably she died, as he then married Christian Langford on 5 January 1759 in Paulton. He was a coalmaster, and after his death on 30 September 1767, in his will (Probate 25 March 1768) he left to Mrs Christian (Kitty) Hill ``indenture land in Paulton... including the engine house, fire engine, outhouses and stables and tools, and things for mining..''.  (The significance of this is that the mine would be worthless without the above ground machinery.) His will also read: ``Will of Said Joseph Hill of Paulton Somerset Coalmaster:  First I do request and hereby authorise my wife Christian Hill if she can within six calendar months next after my decease to buy and purchase of and from my brothers John Hill, Robert Hill and my sister Elizabeth Palmer, widow, all their and each and every of their part of the coalworks..."  The inscription on the family grave vault in Paulton churchyard reads:
``To the Memory of Joseph Hill Esq of this parish, who died the 30th of September 1767 Aged 40 years.
  also of Christian Hill widow of the above Joseph Hill who died 12th day of July 1807 aged 72 years.''
(An indenture document from that time reads: ``Between Joseph Hill of Paulton in the County of Somerset Coal Master of the one part and John Hill Innholder, Robert Hill Butcher and Elizabeth Palmer widow. Bought from James Dando''.
Their third son was \textsc{Robert Hill} who was born in 1731. He married Mary Ames, and was a butcher. He died in 1787. 	
The youngest child was \textsc{Elizabeth Hill}. All that is known of her is that she married a Thomas Palmer in Timsbury, Somerset, and they had at least one child, Mary Palmer.

The next generation descends from \uppercase{John Hill}, who was born in Paulton in 1729; he married \uppercase{Elizabeth Annie Ames} in 1751.  They had nine children: Simon Hill, Joseph Hill, Thomas Ames Hill, Elizabeth Hill, Hepzibah Hill, Elizabeth Hill, Susanna Hill, John Hill, \uppercase{ George Hill}, who was \uppercase{Ralph Munday Denton Barker}'s great-great grandfather, and Robert Hill.
\textsc{Simon Hill} was born in Paulton, Somerset,  and christened on 2 November 1752. He owned the colliery known as``Simon's Hill'' from 1790 to 1800. In 1791 coal was being sold at four pence a bushel from this pit. Simon, who never married, had an illegitimate son, Joshua, by his servant Ann Noel. When Simon died this child was only four years old but his father left money for his upbringing although it is not known what happened to him and his mother in later life. Simons Hill Coal works were bought by a Reverend James Rawlins. Simon died on 3 December 1814 at New House, Paulton.
\textsc{Joseph Hill} was born in 1755 in Paulton, and died on 27 November 1782.
\textsc{Thomas Ames Hill} was born in Paulton, Somerset, and was christened on 11 October 1758. He married Mary Pope but they had no children. He was the innkeeper of the Red Lion in Paulton and died on 18 August 1827.
\textsc{Elizabeth Hill} was born in Paulton and christened on 5 November 1760. She did not marry and died at a young age, on  25 March 1781.
\textsc{Hepzibah Hill} married Jonathon Parsons on 6 March 1789 in Paulton, Somerset.
\textsc{Susanna Hill} was born in Paulton and christened on 26 February 1765. She married John James and they had two sons, Thomas James and John James.
\textsc{John Hill} was born in 1767 and died on2 July 1796. He never married.
\textsc{Robert Hill} was born in 1775 in Paulton, Somerset. In 1839 his occupation was listed as ``Gentleman''. He married Mary Evans, and they had no children. He died on 25 November 1839 and in his will he leaves all his property to John Hill, gentleman, the younger of Paulton, nephew; and requests John Hill to allow Mary to live there until she dies.

In the next generation, \uppercase{John Hill} and \uppercase{Elizabeth Ames}'s ninth child \uppercase{George Hill} was born in Paulton, and he married \uppercase{Hannah Dando} in 1803. They had eleven children: John Hill, James Dando Hill (b.1806, died in infancy), Elizabeth Hill (born in 1807 and died age 22), \uppercase{Mary Hill}, Thomas Hill (b.1810, died in infancy), Susannah James Hill, Anna Maria Hill, Sarah Hill (b.1822,died in infancy), Thomas Ames Hill, Robert Hill (b.1825, died in infancy) and Sarah Ann Hill (b.1826, died young). 
\textsc{John Hill} John Hill was born in 1804 at Paulton House and married Jane Ann Lambert. They had no children. He was the Innkeeper of the Red Lion, (inherited from his uncle) and in 1856 he was known as a 'junior yeoman'; he was also a wine and spirit merchant. He was the Parish Churchwarden for nearly 40 years. His will left everything to his wife, Jane Ann and they lived at Hill House after Hannah, his mother, moved next door to the Silk House. He died on 10 April 1871 and was buried in Paulton churchyard.
\textsc{Susannah James Hill} was born on 4 June 1815 at Hill House, Paulton. She married Henry Thomas Munday on 24 October 1844 and they had one son, Henry Thomas Munday(b.1845). She died on 30 May 1845.
\textsc{Anna Maria Hill}, known always as Maria, was born on 14 March 1817 in Paulton. She married Bruges Fry on 30 October 1839, who was a coroner and magistrate in Cheddar, Somerset and \uppercase{John Hill Munday} lived with them when he was young. She died on 4 December 1869 in Cheddar.  In her will, 'Aunt Maria' leaves 900 pounds to Louisa Fry (who was living with her at the time), and the rest of her money was distributed as follows: ``100 pounds to niece Ann Lawrence wife of George Lawrence, 100 pounds to niece Lucy Ann Frisby wife of James Frisby (gentleman), 100 pounds to Sarah Parsons, and the rest  divided between nephew George Fry Parsons, niece Maria Louisa Parson, niece Mary Catherine Parsons. To brother John Hill the younger, sister Mary Munday and nephew Thomas Henry Munday all other money. Clothes, trinkets and jewellery to Mary Munday. Carriage and horse to Thomas Ames Hill''.
\textsc{Thomas Ames Hill} was born on 18 February 1823 in Paulton, Somerset.  He married Elizabeth Jane Alford on 7 December 1848 and they lived at the Silk House (Glenvue) in Paulton. They did not have children. He was a solicitor and Magistrate, was a landowner, and also had an interest in the Wells Way Coal Works, which he inherited from his uncle John Hill James (this interest was worth 800 pounds with 500 pounds paid in cash). He also had shares in the Paulton Coal Company and the Radstock and Wells Way Coal Works.  He was a well to do Solicitor and Magistrate.  He died on 26 January 1894 of Bronchitis and pulmonary congestion.
















