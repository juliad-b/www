\section{Barker}
The first generation of the Barker family started with Charles Frederick Barker, who was born in Copenhagen in 1801  and would not have been actually called "Barker" at that time (he was Ralph Munday Denton Barker's great grandfather). He married  Elizabeth Hazelwood (1807 - 1882) in 1836 and they had four children,  Charles Frederick Barker (1836 - 1887), Elizabeth Barker (1838 - died as infant), Thomas Henry Barker (1841 - 1917) who was Ralph's grandfather, and Joseph Bolton Barker (1844 - ?).

Their eldest son,  Charles Frederick Barker was born on 30 November 1836 in Stepney, Middlesex, and baptised on 3 October 1837 at St. Dunstans in Stepney. On 22 April 1869 in Everton, Lancashire, he was listed as being a Full Mariner. He married Isabella Fearon on 22 April 1869 at St.Augustine's church,	Everton, Lancashire, but  she died in April 1878, and he then married Barbara Lisle Charlton on	6 May 1879 at	Monk-Wearmouth, Durham.  He and Isabella had two children, Hilda Blanche Barker ( born in 1873, she married Samuel Harrison in Southport, Lancashire), and Charles Frederick Fearon Barker (Born in 1878, he was a soldier at Ladysmith, Natal, South Africa and then emigrated to British Columbia in 1903). Isabella died in April 1878 and he then married Barbara Lisle Charlton on 6 May 1897 at Monk-Wearmouth, Durham. They had two children, Charles Lisle Strangways Barker (born in 1880, he emigrated on 12 March 1906 to	Boston, Suffolk, Massachusetts, United States on the Cymric and perhaps worked as a bank clerk, but returned to England and died in 1929 in Canterbury) and Charles Gordon Cooper Barker (born in 1882, he too emigrated on 3 November 1909 to	Sweetgrass, Toole, Montana, United States where he married Marie Tunelins in August 1913 in Phillips, Montana; later working as a cook for a Chicago dairy company).
Charles was an apprentice seaman, on his first voyage, on the Ranee in 1853 on the England-India route; his father was the Master but died at sea on that same voyage. He was a Master himself by 1868, of the Scotia on the US trade route, and then the Australasian (US again). He was examined in London for his Masters Certificate of Competency in 1867, but by 1871 he is no longer listed in Lloyds Captains Registers so probably had left the sea by then. He died on 22 December, 1887,  in Liverpool and is buried at Anfield Cemetery.



