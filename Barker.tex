\section{BARKER}

\textbf{First Generation}

This family line started with Charles Frederick Barker, who was born in Copenhagen in 1801 and would not have been actually called ``Barker'' at that time (he was \uppercase{Ralph Munday Denton Barker}'s great-grandfather). He married  \uppercase{Elizabeth Hazelwood} in 1836 and they had four children,  Charles Frederick Barker, Elizabeth Barker (1838 - died as infant), Thomas Henry Barker (who was Ralph's grandfather), and Joseph Bolton Barker.

Their eldest son,  \textsc{Charles Frederick Barker} was born on 30 November 1836 in Stepney, Middlesex, and baptised on 3 October 1837 at St. Dunstans in Stepney. On 22 April 1869 in Everton, Lancashire, he was listed as being a Full Mariner. He married Isabella Fearon on 22 April 1869 at St.Augustine's church,	Everton, Lancashire; he and Isabella had two children, Hilda Blanche Barker ( born in 1873, she married Samuel Harrison in Southport, Lancashire), and Charles Frederick Fearon Barker (born in 1878, he was a soldier at Ladysmith, Natal, South Africa and then emigrated to British Columbia in 1903). Isabella died in April 1878 and Charles then married Barbara Lisle Charlton on 6 May 1897 at Monk-Wearmouth, Durham. They had two children, Charles Lisle Strangways Barker (born in 1880, he emigrated on 12 March 1906 to	Boston, Suffolk, Massachusetts, United States on the Cymric and perhaps worked as a bank clerk, but returned to England, calling himself a rancher, and died in 1929 in Canterbury) and Charles Gordon Cooper Barker (born in 1882, he too emigrated on 3 November 1909 to	Sweetgrass, Toole, Montana, United States where he married Marie Tunelins in August 1913 in Phillips, Montana; later working as a cook for a Chicago dairy company).
Charles was an apprentice seaman: on his first voyage on the \textit{Ranee} in 1853, sailing from India back to Britain, his father was the Master and died at sea on that same voyage. He was a Master himself by 1868, on the \textit{Scotia} on the US trade route, and then the \textit{Australasian} (US again). He was examined in London for his Masters Certificate of Competency in 1867, but by 1871 he is no longer listed in Lloyds Captains Registers so probably had left the sea by then. He died on 22 December, 1887,  in Liverpool and is buried at Anfield Cemetery.

Their third son, \textsc{Joseph Bolton Barker} was born towards the end of 1844 in Liverpool. He worked as a merchants clerk at Lloyds in Liverpool, and married Amelia Jane Day in September 1873 in the Wirral before moving south, living in Streatham, Surrey. He worked for Lloyds until retiring in 1901 (his date of death is unknown). They had four children. Their oldest son was Henry Charles Day Barker: born in 1875, he became a naval captain and JP, living in Cornwall; he married Margaret Bunning Smith in 1900. Their second child, Ida Day Barker,  was born in 1877 in Streatham, Surrey, and probably never married: she may well have joined her younger brother Percy in Pingelly, Western Australia for some time, as her name appears on property deeds in Western Australia. Later in her life she lived at Grey Wethers, Ivybridge, Devon. Their third son, Percy Strangeways Day Barker was born in 1879 in Peckham, Surrey, and he emigrated to Western Australia; in 1910 he lived  on his farm, Nettadyne, at East Pingelly; he farmed the land (approximately 2000 acres) until the beginning of the Second World War, and was responsible for clearing a great deal of the old large timber. Nettadyne is on rolling, granite country and apparently had some of the largest trees in that area before clearing. He was known in the district for his fine turnout of horses and was considered a good farmer. He never married. In 1945 the farm was requisitioned by the government (possibly for returned soldiers) and he retired to Albany where he died in 1960. Their youngest son was Frederick Day Barker, born in 1879 in	Streatham, Surrey. There is no record that he married. He became an accountant and is recorded as arriving in Plymouth on the \textit{Abosso} in 1915, having come from Lagos, Nigeria where he had been on Government Service and again in 1916 on the \textit{Apapa} and 1920 on the \textit{Appam} when he gave the Road Club, Coventry Street, London as his UK adddress (date of his death unknown).



