\section{Munday}

The first Munday family member we know about lived in Wiltshire at Bishopstrowe, near Warminster.
James Munday married Jemima Browne, and they had nine children, one of whom was William Munday (who is listed in "CFB").  His two brothers James and John went to Australia in 1830 - they were hoping  to find land and settle there.  
The oral history as  recorded by John Hill Munday (see "CFB") from his aunts Kate and Elizabeth Munday (sisters to John, James and William) is as follows: "John Munday, son of James and Jemima, and James emigrated to the Swan River which was at that time considered to be the most promising of the fields of emigration. They took with them farming implements and a labourer. James turned his attention to building ... John however being disgusted with the misfortunes encountered at Swan River migrated to Hobart town, where some mystery envelops his career. It is supposed however that he became entangled with sharpers, as he ultimately wrote home for the share of money due to him under his father's will. This was sent out but what happened after this is not known. He however disappeared and it is supposed committed suicide. This happened in 1835. When the Rev. K. Thorpe went over to Hobart Town in 1861 he made enquiries on the subject but no light was thrown on it, as probably the colony was in too unsettled a condition that no regular government existed at that time."

In fact, John and his brother James arrived in the Swan River Colony on July 6 1830 (on the Medina, captained by Capt. Pace). The following year they travelled on to Van Diemens Land on the Eagle, arriving in Hobart Town on 11th January 1831. (James, however, soon returned to the Swan River Colony). John was living in Launceston by early 1835. (His activities and whereabouts for the previous few years are not known.) He had been granted the licence to run the public house, the Joiner's Arms, which was on the corner of George and Brisbane Street, previously owned and run by David Williams from 1832.(N1,S4)

John died in Launceston on 12 February 1835, only two days after he had been granted the licence, and he did in fact commit suicide. The original Inquest documents are held at the Tasmanian State Archives and they read as follows:

"An Inquisition taken for our sovereign Lord at the Parish of Launceston in the county of Cornwall the fourteenth day of February in the fifth year of our sovereign William the Fourth by the grace of God on the United Kingdom of Great Britain and Ireland King defender of the Faith before Peter Archer Mulgrave Esquire one of the coroners of our said Lord the King for the said county on view of the body of John Munday, then and there lying dead upon the oath of William Milne, Richard White, William Gilbert, Richard Ruffin, William Bullock, Thomas Symons, John Ashton, Robert Brand, Joseph Dell, Jeremiah R?, George Lucas, all good and lawful men of the said county duly chosen and who being then and there duly sworn in and being charged to inquire for our said Lord the King when where how and after what manner John Munday came to his death do upon their oath say that the said John Munday not being of sound mind memory and understanding on the twelveth day of February in the year aforesaid at the parish and in the dwelling house of David Williams did there with a certain razor made of iron and steel which he then laid then and there had and held in his right hand at the throat of him the said John Munday did thrice stabb (sic) and penetrate with the razor aforesaid, and inflicted one mortal wound of the length of four inches and of the depth of four inches of which said mortal then laid John Munday then and there instantly died. And the jurors aforesaid do sign their oath. (followed by their signatures)

The following are statements by four witnesses called to the inquest:

Mr Clark: I have resided at David William's house since September, and since three weeks, John Munday has lived at William's house. He was in reduced circumstances and occasionally very low spirited during that time. He borrowed four and twenty pounds off William for the purpose of paying for a licence to keep a public house which he recently obtained. Abut four days since he received a letter from England which he said contained indifferent news and caused him to shed many a tear and he very much regretted that he had ever left home which that time he had been low spirited. At one o'clock yesterday I was sitting with William in the parlour when Munday passed through the room into William's bedroom where the body is now lying, he shut the door. I did not see that he had anything in his hand as he passed through the room. After he had been in the room about ten minutes I heard a noise resembling a ? and asked William if he knows if Mr Munday had taken anything, he said he did not. William went to the door and tried to open it. It was locked, he did not knock the door or call out but ran to the window of the bedroom at the front of the house and I saw the deceased lying with his head upon his arms under the window and saw a spot of blood on his sleeve. William lifted up the sash and we then saw blood upon the floor and then went into the passage and called the cook "Hurry" and told him to contact Mr.? and breaking open the bedroom door looked in. He did not speak to Williams or myself before he passed through the room. Munday had a small box in Mr William's room and he was in the habit of going in to change his clothes. He was lying on the floor with this razor made of iron and steel lying open bent back. I immediately ran to fetch a doctor and returned with him to the house within six or seven minutes after I had left it. The body was then lying in the same state as when I last saw it. Doctor L? examined the wound and had the body lifted up onto the bed. I never heard Munday after I saw him leaning with his head upon the table under the window. There was no person in the bedroom beside himself. this razor belongs to Mr Williams and is usually dept on the dressing table under the window in his bedroom. He had not been drinking yesterday or the day before or quarelling with anyone that I know of (signed) J.B. Clark

Mr David Williams: The deceased had been living in my house about three weeks. I let him my house about a fortnight ago for a public house and lent him four and twenty pounds to pay for a licence. I know at the time he was in distressed circumstances but believed he was respectable and would obtain money from Mr ?. Last Saturday he received a letter by post which he said contained bad news and caused him many a tear. He frequently said he wished he had never left home. I have not observed him particularly since last Saturday. A few moments before I was siting in the parlour conversing with Mr Clark when Mr Munday walked through into my bedroom where he was in the habit of changing his clothes. He locked the door after him which I thought somewhat strange and afterwards I heard something dropping on the floor and an odd noise.. Mr Clark said has Munday been taking anything this morning I said I do not know he said you had better see I went to the bedroom door and found as I supposed that it was locked. I said "Munday" but received no answer. Mr Clark and myself ran to the window of my bedroom at the front of the house and threw up the sash and saw him lying with his head upon his hands under the window. Mr Clark said there is blood upon his sleeve I said he has cut his throat Mr Clark and I ran into the passage and called the cook and he came and ? breaking open the door. There was blood flowing from a wound in his throat.

James Hannaway (cook/servant): I have known John Munday about a fortnight. He was generally a very early riser. He appeared occasionally very low spirited and a very close man. On Thursday morning last he stayed in bed and then in the course of the afternoon he asked me to shave him he said he trusted me because he frequently cut himself. I saw nothing remarkable in his manner, about one o'clock I was in the kitchen and heard my master and Mr Clark call to me my master said lend a hand there is something the matter in this bedroom and desired help in forcing open the door which was locked on the inside. We forced it open I then saw John Munday lying on his right side and blood flowing from a wound in his throat and a razor lying near his right shoulder. Mr Clark went to the door and sent for a doctor. The body had not been moved from the time we broke open the door until the Doctor arrived.

From Mr J Dudley : This is the letter alluded to by Mr Williams it appears to have been written by a sister of the deceased and appears calculated to cause despondency in the mind of a person whose circumstances were embarrassed. He owed two notes one for 102 pounds and the other for 40 pounds in his handwriting, I believe he has been in low circumstances lately, I knew nothing of his money transactions. I know that he expected a good deal of money at Swan River and recently here, and that the loss of money has prayed (sic) upon his mind and I have heard him frequently regret his ever having left home. I saw him one day between Saturday and Monday when he appeared more low spirited than usual.

This information on oath of William Lecount? Esquire, Assistant Colonial Surgeon in Launceston as follows: I was called upon about one o'clock on Thursday last to see a man at the house lately occupied by Mr David Williams. I went into a bedroom on the right hand side of the passage and saw John Munday lying upon the floor by the side of the bed with his arms extended and his head towards the end of the bed. A razor was lying by his side open and covered with blood. There was a quantity of blood upon the floor. He appeared to be dying. The wound had divided the left jugular and partially the left artery. He died between 5 and 10 minutes from the effects of that wound.

From Mr John Biles: I knew the deceased John Munday intimately. I saw him on last Wednesday he showed me a letter which he had received from his sister by which he appeared a good deal excited. He did not appear in low spirits but I think that his appearance of good spirits was forced. I do not know if his circumstances were embarrassed or not. \footnote{Inquest}

    ↑ Tasmanian State Archives: Record of Inquest 1835.

  

    Place: Bishopstrow, Wiltshire, England; ; Date Range: 1674 - 1890; Film Number: 1279445.
      "Licensed Victuallers of Van Diemens Land", Launceston Library local studies Collection.

    Hobart Town Gazette, Thursday February 12, 1835, p.118, c.2-3
      Swan River Letters, Vol 1 - Ian Berryman p.178.

    "Arrived per 'Medina' on 6 July 1830 with his brother John. Requested permission to leave the colony on 17 July 1830 (Govt. notices 1393) and they both left per 'Eagle' for Van Diemen's Land in August 1830."
      'Launceston Advertiser', February 19th 1835, P.3.

    Accessed Hobart Library, microfilm, Launceston Advertiser, 9 February - 31 December 1835
      Colonial Times, Hobart Town Feb. 4th 1831, P.2.
      Notice No.11 -
    Internal Revenue Office, 10th February 1835
    The undermentioned parties residing in the division of the Island of Van Diemen's Land, commonly called Buckinghamshire and Cornwall, having applied for and obtained a certificate of approval:
    Buckinghamshire -
    William Gordon Nolan, "The Morning Star" Hobart Town
    Cornwall -
    Alexander Waddle, "The Blue Bell" Launceston
    John Munday, "Joiner's Arms" ditto
    Robert Stonehouse, "Crown" ditto
    Samuel Sherlock, "White Hart" George Town
      "A young man named Munday, residing in Launceston, committed suicide by cutting his throat in the most dreadful manner, one day last week. So effectually (sic) did the unfortunate man accomplish his purpose that his head was nearly severed from his body. No particular cause could be assigned for the rash act. The Jury at the Inquest on the body returned a verdict of Temporary Insanity." (S6)
      "The Eagle schooner arrived in Hobart Town on Sunday night, bringing 72 passengers from Swan River. Double and treble that number would have been embarked had there been room on board. The accounts are lamentable - prospects become daily worse. A report is in circulation that orders are sent, via Hobart Town, to break up the settlement and to transfer the whole population."


JOHN MUNDAY

    F.  James Munday1760 - 1827
    M.  Jemima Browne1770 - 1839

m. 15 Feb. 1798

    Jemima Munday1798 - 1870
    William Munday1800 - 1886
    Catherine Munday1802 - 1883
    Sarah Munday1803 - 1869
    James Munday1805 - 1863
    Mary Elizabeth Munday1807 - 1896
    John Munday1809 - 1835
    Henry Thomas Munday1813 - 1895
    George Munday1815 - 1830
