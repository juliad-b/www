\section{Munday}

The first Munday family member that we have on record was James Munday, who was born in Bishopstrowe (near Warminster in Wiltshire) in 1760.  He  married Jemima Browne, and they had nine children, one of whom was William Munday (who was Ralph Munday Denton Barker's great grandfather). Nothing is known of six of the children, but two brothers, James (1805--1863) and John (1809--1835), went to Australia in 1830 - they were hoping  to find land in the Swan River Colony and settle there. 

The oral history as recorded by John Hill Munday (R.M.D-B's grandfather) from his aunts Kate and Elizabeth Munday (sisters to John, James and William) with regard to the two brothers is as follows: 

\begin{quotation}
John Munday, son of James and Jemima, and James emigrated to the Swan River which was at that time considered to be the most promising of the fields of emigration. They took with them farming implements and a labourer. James turned his attention to building ... John however being disgusted with the misfortunes encountered at Swan River migrated to Hobart town, where some mystery envelops his career. It is supposed however that he became entangled with sharpers, as he ultimately wrote home for the share of money due to him under his father's will. This was sent out but what happened after this is not known. He however disappeared and it is supposed committed suicide. This happened in 1835. When the Rev. K. Thorpe went over to Hobart Town in 1861 he made enquiries on the subject but no light was thrown on it, as probably the colony was in too unsettled a condition that no regular government existed at that time.
\end{quotation}

In fact, John and his brother James arrived in the Swan River Colony on July 6 1830 on the Medina, as that colony was considered to be the most promising of the fields of emigration at that time; the first immigrants  were attracted by glowing reports of a fertile land suitable for agriculture and opportunities to make their fortunes. Advertisements and placards had been posted all over England, and newspapers published feature columns on these opportunities. Contemporary observers called it `Swan River mania'. The brothers took with them farming implements, and James received a grant of land in the newly surveyed town of Kelmscott where he was granted 15 acres of land on the bank of the Canning River, but he appears not to have taken that up for long (emigrants who arrived after 1830 were given 20 acres of land for every three pounds of capital invested, so clearly James did not have much money). They travelled on to Van Diemens Land on the Eagle, having requested permission to leave the colony on 17 July, and arriving in Hobart Town on 11 January 1831. Things were not going well in the West and this was clear from the Hobart Town Gazette, which wrote: `The Eagle schooner arrived in Hobart Town on Sunday night, bringing 72 passengers from Swan River. Double and treble that number would have been embarked had there been room on board. The accounts are lamentable - prospects become daily worse. A report is in circulation that orders are sent, via Hobart Town, to break up the settlement and to transfer the whole population.'  James had written a long letter to the Editor of the Colonial Times and Tasmanian Advertiser in Hobart, which was printed on 29 June 1831 and read as follows: 

\begin{quotation}
To the Editor of the Colonial Times.

Having read a statement in the Hobart Town Courier of the 5th February relative to the Swan River, furnished by `a passenger by the Eagle' and conceiving that the worthy Editor would not consciously become the promulgator of misrepresentation or untruth, I request your insertion of a refutation of that very unfounded statement, considering it to be duty of every settler in this rapidly improving colony to contradict the false reports which have been made, either by persons to answer their private ends, or those whose mismanagement or misconducts have obliged them to leave the settlement. The informant states that Kelmscott is a desirable situation for a town, and enumerates some of its local advantages - so far he is correct; but he proceeds to state `that some settlers went up there, but left the place in disgust, owing to the ill management of the person appointed to locate them; he fixing them in one place, and after they had gone to considerable expense informing them that they most remove, for the situation they occupied was a government reserve, the result was that they had come on to Hobart town'. This statement is wholly untrue - but knowing the circumstances upon which his Correspondent pretends to found his assertions, I will state the facts. Mr James Munday being desirous of obtaining a grant at Kelmscott, moved up into the neighbourhood before the town survey was completed, and on His Excellency the Lieutenant Governor visiting that part of the country he permitted Mr Munday to select a location, on the same day giving out a notice that persons who for their accommodation were allowed to choose their allotments before the town was surveyed, were to consider that they were liable to have their boundary line moved if the Lieutenant Governor should think fit, on the plan of the town being completed. On the surveyor finishing his map, which he neglected to do for a considerable time, it appeared that three chains of the ground Mr Munday had selected, at the northern extremity of his grant, were marked as a Government reserve for a bridge. Mr Munday was apprised of this circumstance within one hour after the map was put into the hands of the Resident and informed of the probability of his having to take three chains of the south instead. He had not gone to any expense on his grant except a temporary rush hut, and none whatever on his reserve. Mr James Munday is still in this colony and has notified his wish to retain his grant, and his intention to improve on it, regretting that he allowed himself to be influenced by the bad example and bad advice of some of our experts. When he does proceed to occupy his grant he will no doubt experience that encouragement and protection which it is the anxious wish of His Excellency to attend to every well conducted settler, and which example those to whom he delegates authority must follow if they attend to their instructions.

The `OFFICIAL OFFICES' as the Correspondent of the Hobart Town Courier terms them, are not removed to Freemantle. Neither is it the intention of the principal settlers, as far as I can learn, to petition the Home Government to make this a penal settlement. We have a pledge that it shall not become so, and we have the satisfaction to know that what we may lack in numbers, we have in respectability. It is not a fact that the expense of breaking and clearing ground is 30 pounds per acre - the expenditure has not exceeded 7-8 pounds at the most expensive time to those who had their own labourers; and at the present rate of provisions land my be broke for 5 pounds an acre.

WE have now surmounted the principal difficulties. The Colony is abundantly supplied with the necessaries of life at moderate prices; and the results of the late expedition have opened up an ample field for the agriculturists and proprietors of stock.

I am, Sir, and co.
One of the Starving Settlers of Western Australia
Kelmscott, Western Australia, May 15 1831
\end{quotation}

James, however, soon returned to the Swan River Colony, leaving John behind in Van Diemens Land. Nothing is known of John's movements until early 1835, by which time he was living in Launceston. He had been granted the licence to run the public house, the Joiner's Arms, which was on the corner of George and Brisbane Street, previously owned and run by David Williams from 1832. The details of this grant was recorded in Notice No.11 - as follows:

\begin{quotation}
Internal Revenue Office, 10th February 1835
The undermentioned parties residing in the division of the Island of Van Diemen's Land, commonly called Buckinghamshire and Cornwall, having applied for and obtained a certificate of approval:
Buckinghamshire -
William Gordon Nolan, ``The Morning Star'' Hobart Town
Cornwall -
Alexander Waddle, ``The Blue Bell'' Launceston
John Munday, ``Joiner's Arms'' ditto
Robert Stonehouse, ``Crown'' ditto
Samuel Sherlock, ``White Hart'' George Town
\end{quotation}

John died in Launceston on 12 February 1835, only two days after he had been granted the licence, and he did in fact commit suicide. The original Inquest documents are held at the Tasmanian State Archives in Hobart, and they read as follows:

\begin{quotation}
An Inquisition taken for our sovereign Lord at the Parish of Launceston in the county of Cornwall the fourteenth day of February in the fifth year of our sovereign William the Fourth by the grace of God on the United Kingdom of Great Britain and Ireland King defender of the Faith before Peter Archer Mulgrave Esquire one of the coroners of our said Lord the King for the said county on view of the body of John Munday, then and there lying dead upon the oath of William Milne, Richard White, William Gilbert, Richard Ruffin, William Bullock, Thomas Symons, John Ashton, Robert Brand, Joseph Dell, Jeremiah R?, George Lucas, all good and lawful men of the said county duly chosen and who being then and there duly sworn in and being charged to inquire for our said Lord the King when where how and after what manner John Munday came to his death do upon their oath say that the said John Munday not being of sound mind memory and understanding on the twelveth day of February in the year aforesaid at the parish and in the dwelling house of David Williams did there with a certain razor made of iron and steel which he then laid then and there had and held in his right hand at the throat of him the said John Munday did thrice stabb (sic) and penetrate with the razor aforesaid, and inflicted one mortal wound of the length of four inches and of the depth of four inches of which said mortal then laid John Munday then and there instantly died. And the jurors aforesaid do sign their oath. (followed by their signatures)

The following are statements by four witnesses called to the inquest:

Mr Clark: I have resided at David William's house since September, and since three weeks, John Munday has lived at William's house. He was in reduced circumstances and occasionally very low spirited during that time. He borrowed four and twenty pounds off William for the purpose of paying for a licence to keep a public house which he recently obtained. Abut four days since he received a letter from England which he said contained indifferent news and caused him to shed many a tear and he very much regretted that he had ever left home which that time he had been low spirited. At one o'clock yesterday I was sitting with William in the parlour when Munday passed through the room into William's bedroom where the body is now lying, he shut the door. I did not see that he had anything in his hand as he passed through the room. After he had been in the room about ten minutes I heard a noise resembling a ? and asked William if he knows if Mr Munday had taken anything, he said he did not. William went to the door and tried to open it. It was locked, he did not knock the door or call out but ran to the window of the bedroom at the front of the house and I saw the deceased lying with his head upon his arms under the window and saw a spot of blood on his sleeve. William lifted up the sash and we then saw blood upon the floor and then went into the passage and called the cook "Hurry" and told him to contact Mr.? and breaking open the bedroom door looked in. He did not speak to Williams or myself before he passed through the room. Munday had a small box in Mr William's room and he was in the habit of going in to change his clothes. He was lying on the floor with this razor made of iron and steel lying open bent back. I immediately ran to fetch a doctor and returned with him to the house within six or seven minutes after I had left it. The body was then lying in the same state as when I last saw it. Doctor L? examined the wound and had the body lifted up onto the bed. I never heard Munday after I saw him leaning with his head upon the table under the window. There was no person in the bedroom beside himself. this razor belongs to Mr Williams and is usually dept on the dressing table under the window in his bedroom. He had not been drinking yesterday or the day before or quarelling with anyone that I know of (signed) J.B. Clark

Mr David Williams: The deceased had been living in my house about three weeks. I let him my house about a fortnight ago for a public house and lent him four and twenty pounds to pay for a licence. I know at the time he was in distressed circumstances but believed he was respectable and would obtain money from Mr ?. Last Saturday he received a letter by post which he said contained bad news and caused him many a tear. He frequently said he wished he had never left home. I have not observed him particularly since last Saturday. A few moments before I was siting in the parlour conversing with Mr Clark when Mr Munday walked through into my bedroom where he was in the habit of changing his clothes. He locked the door after him which I thought somewhat strange and afterwards I heard something dropping on the floor and an odd noise.. Mr Clark said has Munday been taking anything this morning I said I do not know he said you had better see I went to the bedroom door and found as I supposed that it was locked. I said `Munday' but received no answer. Mr Clark and myself ran to the window of my bedroom at the front of the house and threw up the sash and saw him lying with his head upon his hands under the window. Mr Clark said there is blood upon his sleeve I said he has cut his throat Mr Clark and I ran into the passage and called the cook and he came and ? breaking open the door. There was blood flowing from a wound in his throat.

James Hannaway (cook/servant): I have known John Munday about a fortnight. He was generally a very early riser. He appeared occasionally very low spirited and a very close man. On Thursday morning last he stayed in bed and then in the course of the afternoon he asked me to shave him he said he trusted me because he frequently cut himself. I saw nothing remarkable in his manner, about one o'clock I was in the kitchen and heard my master and Mr Clark call to me my master said lend a hand there is something the matter in this bedroom and desired help in forcing open the door which was locked on the inside. We forced it open I then saw John Munday lying on his right side and blood flowing from a wound in his throat and a razor lying near his right shoulder. Mr Clark went to the door and sent for a doctor. The body had not been moved from the time we broke open the door until the Doctor arrived.

From Mr J Dudley : This is the letter alluded to by Mr Williams it appears to have been written by a sister of the deceased and appears calculated to cause despondency in the mind of a person whose circumstances were embarrassed. He owed two notes one for 102 pounds and the other for 40 pounds in his handwriting, I believe he has been in low circumstances lately, I knew nothing of his money transactions. I know that he expected a good deal of money at Swan River and recently here, and that the loss of money has prayed (sic) upon his mind and I have heard him frequently regret his ever having left home. I saw him one day between Saturday and Monday when he appeared more low spirited than usual.

This information on oath of William Lecount? Esquire, Assistant Colonial Surgeon in Launceston as follows: I was called upon about one o'clock on Thursday last to see a man at the house lately occupied by Mr David Williams. I went into a bedroom on the right hand side of the passage and saw John Munday lying upon the floor by the side of the bed with his arms extended and his head towards the end of the bed. A razor was lying by his side open and covered with blood. There was a quantity of blood upon the floor. He appeared to be dying. The wound had divided the left jugular and partially the left artery. He died between 5 and 10 minutes from the effects of that wound.

From Mr John Biles: I knew the deceased John Munday intimately. I saw him on last Wednesday he showed me a letter which he had received from his sister by which he appeared a good deal excited. He did not appear in low spirits but I think that his appearance of good spirits was forced. I do not know if his circumstances were embarrassed or not. \footnote{Tasmanian State Archives: Record of Inquest 1835}
\end{quotation}

The `Launceston Advertiser', (February 19th 1835, P.3.) carried the story:
\begin{quotation}
A young man named Munday, residing in Launceston, committed suicide by cutting his throat in the most dreadful manner, one day last week. So effectually (sic) did the unfortunate man accomplish his purpose that his head was nearly severed from his body. No particular cause could be assigned for the rash act. The Jury at the Inquest on the body returned a verdict of Temporary Insanity.
\end{quotation}

James, meanwhile, had returned to the Swan River Colony and then turned his attention to building, entering into partnership with James Woodley Davey in the newly gazetted town of Fremantle. He was listed as a carpenter — in the 1832 census as `single, age 25, carpenter, b. Wilts, came on the Medina' \footnote{Dictionary of Western Australians 1829-1914, Vol. 1 Early Settlers 1829-1850, UWA Press 1979, p.244.}. The two Jameses were granted Fremantle building lots S527 and S538 on 23 September 1833. They also had Lot No.40, at the end of Mouat Street (next to the Watermans Arms), Lot No.20 in Pakenham Street, and Lot No.93, at the southern end of Henry Street. They also had some land as a `timber allotment' two miles from Fremantle on the Bull Creek Road. In 1831, Mary Ann Friend (who was also on the Medina) wrote of Fremantle: `Never slept in such a miserable place; everything so dirty, sheets etc. such quantities of mosquitoes and fleas...'  Life would have been hard for James, with many people still living under canvas without sufficient tools and supplies. 

Davey took over the partnership, which was subsequently dissolved, in 1835, when James went back to England in 1835. The documentation of this were as follows:
\begin{quotation}
"Date of Registration: 11 April 1835
Date of original Document: 14 March 1835
A power of attorney to empower James Davey to manage the joint property of James Munday and the said James Davey; to make improvements theron to be paid from James Munday's share therein, with power to sell or exchange the same. Parties James Munday of Fremantle, Carpenter to and in favor of James Davey of Fremantle, Carpenter.
Description of the Land:
All those Town allotments of land in Fremantle and buildings thereon being
Nos. 93 adjoining McDermotts,
120 and 121 in Pakenham Street occupied by Mr Okeley and
40 in Mouat Street occupied by Mr Steele.
\end{quotation}

James married Sophia Davis on 28 July 1836 in	Walcot, Somerset.  She refused to go back to Australia, so he lost his business there. They then went to Montreal, but found the climate `too fierce', and returned to live in Cheltenham and then Worthing, then London where he died in 1863.

In the next generation, William Munday and Mary Hill had ten children, one of whom was John Hill Munday (see CFB). The oldest son was George Munday (born on 17 October 1836) and he died after a fall from a horse in Jamaica on 16 March 1862. A note held in the family reads: `29 June 1852 George Munday got to Jamaica after a short passage of 5 weeks', and a letter from John Hill Munday to his sister Anna reads:

\begin{quotation}
Cheddar April 14th 1862
In haste. Uncle has called me away 3 or 4 times since I began.

My dearest Anna,
   I am sorry to be the bearer of very melancholy sad intelligence. Poor George is no more! The Jamaica Mail brought the sad intelligence this morning. Uncle received a letter from Mr. Kitc?son this morning stating that poor George was out riding on the 9th Mar. last with Walter Thorn, & another young man. It was rather late and they were riding a hard canter but not galloping, when as they were turning a corner, Geo's horse stepped on a flat stone and slipped down, George falling with him, Mr. Thorn instantly got off an lifted him up and saw blood coming from his nose & mouth he laid his head on his knee and sent the other young man for some water & put a hankerchief to his nose to catch the blood. The young man soon came back & said he could not get any water, the hankerchief was then completely saturated with blood. He then sent him to another house where he succeeded in obtaining some & brought several people to their assistance. They then sent for the Dr. who came as quickly as possible and immediately pronounced the case to be fatal.
    The poor fellow never spoke after it happened & was quite senseless. You may imagine how shocked and grieved we all are. It has quite upset me.
    How little did I think when spending such a happy day with you yesterday that I should be plunged into such sorrow toda but we must try to bear it & hope to meet him again where we shall never part. I sincerely hope that he was prepared to die. It was just 3 weeks after Wm. Parsons funeral. His prospects were just brightening & everything seemed prosperous & well with him but God saw best to take him & we must not repine. I expect Mother heard the sad news from Walter Thorn & Hy. Parson this morning & you will no doubt hear from her. I have told you all the particulars I know.
    Aunt Maria sends her love to you & to say that she has heard from Mrs. Simmons about your going there at Easter & thinks that you had better accept it as the change will do you good.
    I got home all right last night within a minute of my time.
    S?nip Fry & Loui desire their kind love & accept the same -
    from your ever affect.dt Brother
    J.H.Munday
    \end{quotation}




 
