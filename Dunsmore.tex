\section{DUNSMORE/DUNSMUIR}

The first person in this line that we know of is \uppercase{Robert Dunsmore}, who was born in 1780 in North Ayrshire.  By	
17 April 1808 he was living at Craighouse, Old Cumnock, Ayrshire, and had married \uppercase{Jean Kirkland}.  They had five children: James Dunsmuir, Marian Dunsmuir, Allan Dunsmuir, Mary Dunsmore who died as an infant, and  their youngest, \uppercase{Jean Dunsmuir}, who was  \uppercase{Joan Nyria Hancox}'s great grandmother.

Their oldest son, \textsc{James Dunsmuir} (note the change in the spelling of Dunsmore to Dunsmuir at this time), was born in	1805 in Riccarton, Ayrshire. He married Elizabeth Hamilton, b.1804 - 1832, and they had four children, Robert Dunsmuir, Jean Dunsmuir, Elizabeth Love Dunsmuir and Marion Dunsmuir. He died of cholera on 18 August 1832 at the same time as his mother, his wife and two of their daughters; they were buried in Riccarton Churchyard (Their Grave inscription in Riccarton Churchyard reads: ``James Dusnmore, late coalmaster, Barleith, d 18.8.1832, a 27y. w Elizabeth Hamilton d 13.8.1832, a 28y, ch Marion \& Elizabeth D d inf.'' Their oldest son, Robert Dunsmuir (1825-1889) had a longer life and was to become famous in British Colombia.  When he was born, the Dunsmores were involved in coalmining in Ayrshire. Dunsmuir's grandfather, \uppercase{Robert Dunsmore}, had leased coal properties and bought out local competitors in the days before the arrival of the railway in the 1840s permitting him to increase prices.  Three years after the cholera epidemic, his grandfather Robert died a relatively wealthy man, leaving a third of his estate in trust for his orphaned grandchildren. James was schooled locally at the Kilmarnock Academy and then at the Paisley Mercantile and Mechanical School, a training helpful in the coal business. He then went to work in local coal mines under his uncle \uppercase{Boyd Gilmour}.  On September 11, 1847, at the age of 22, Robert married 19 year old Joan White. Eight days later, their first child, Elizabeth Hamilton was born. Under the strict rules of the Presbyterian Church, Robert and Joan were required to confess their sin of sex prior to marriage before the whole congregation to have their daughter baptized in the Kirk. Their second child, Agnes, was also born in Scotland in 1849. At the end of 1850 his uncle \uppercase{Boyd Gilmour} signed on with the Hudson's Bay Company to exploit a coal finding on the northeast coast of Vancouver Island at Fort Rupert (near present day Port Hardy). Because some of those who were to travel with him decided not to go upon hearing news of the conditions and prospects there, Gilmour sought replacements for his party at the last moment. On 24 hours' notice of this opportunity, Robert signed on. They sailed on the Pekin, for Fort Vancouver, via Cape Horn, on December 19, 1850. It took 191 days for them to arrive and their third child, James, was born on July 8, 1851, just before they arrived in Oregon. On July 18, 1851 they set sail for Fort Rupert, and when they arrived on August 9, the three-year term on the contract with the Hudson's Bay Company began. Gilmour struggled unsuccessfully to develop a producing coal operation at Fort Rupert until August 24, 1852 when Governor Douglas instructed them to move on to Nanaimo where a coal seam had been discovered. Work proceeded but living conditions were difficult. In 1854 when the term of their HBC contracts came up and Governor Douglas refused to increase their pay rates, Gilmour left to return to Scotland. Dunsmuir stayed on. He went on to propose to Douglas that he carry on personally with the operation of a seam that Gilmour had thought was played out. On October 12, 1855, Dunsmuir commenced work on his own account and within a month was producing seven tons of coal a day. This venture was a modest success, but as the seam ran out, Dunsmuir was again employed to operate a new pit that the HBC opened in 1860.  (According to a census taken in 1854, the white population of Nanaimo was 151. There were 52 dwelling houses, 3 shops, and 1 school with 29 students, including the Dunsmuir children. Dunsmuir impressed James Douglas, the Colonial Governor and Chief Factor of the HBC, as a stable and hardworking man who could be relied on to complete a task with a minimum of trouble. Dunsmuir was given a longterm contract with the HBC’s coal company. He later became one of the wealthiest men in British Columbia and his son James became premier in 1900.)

Their second child, \textsc{Marian Dunsmuir} was born on 17 April 1808 in Old Cumnock, Ayrshire. She married John Dunsmuir on 16 October 1829 in Kilmarnock, Ayrshire. They lived in Dalry and had seven children:  Robert Dunsmuir (b.1830), Jean Dunsmuir (b.1832), John Dunsmuir (b.1834), Margaret Dunsmuir (b.1840), Ann Dunsmuir (b.1843), Mary Dunsmuir (b.1847) and Allan Dunsmuir (b.1848).  Marian died from `supposed heart disease' on 21 June 1872 in Dalry and her son Allan was the informant (only a month before her husband died).

Their second son was \textsc{Allan Dunsmuir}. He married Agnes Grant on 2 July 1824 in Kilmarnock, Ayrshire, at the Low church.  By 1847 they were living in Hurlford, Ayrshire, at Braehead Cottage, and he was a prosperous Coalmaster. They had five children: Robert Dunsmuir (b.1828), Allan Dunsmuir (b.1830), Agnes Dunsmuir (b.1833), Mary Dunsmuir (b.1835) and Marion Dunsmuir (b.1837). He died on 13 July 1847 in Hurlford.

  












